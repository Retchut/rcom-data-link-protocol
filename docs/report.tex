\documentclass[11pt]{article}

\usepackage{graphicx}
\usepackage[utf8]{inputenc}
\graphicspath{{./images/}}

\title{\includegraphics[scale=0.3]{logo.png} \\ \textbf{Data Link Protocol}}
\author{Computer Networks\\ Bachelors in Informatics and Computing Engineering \\ \\ 3LEIC03\_G3 \\ \\ Tiago Rodrigues up201907021@fe.up.pt \\ Mário Travassos upidk@fe.up.pt  }
\date{\today}

\begin{document}

\maketitle

\newpage

\section*{Summary}

\paragraph{}This report will cover the first work proposed for the Computer Networks Curricular Unit, with the objective of creating a small application that could transfer data through two computers asynchronously, through a serial port.

\paragraph{}The application is capable of transferring files whilst mainting their integrity, and detect errors in transmission, resolving them if possible.

\section*{Introduction}

\paragraph{}This report is the result of an examination to the practical component, which was the development of a data transfer protocol. A serial port was used to transfer the files in an asynchronous fashion.

The report is organized as follows:

\begin{enumerate}
  \item{Architecture- Functional blocks and interfaces}
  \item{Code Structure - APIs, main code structures and their relation with the architecture}
  \item{Main use cases - Identification and Call Stack Sequence}
  \item{Data link Protocol - Main functional aspects and implementation strategy}
  \item{Application Protocol - Main functional aspects and implementation strategy }
  \item{Validation - Description of the tests conducted}
  \item{Efficiency - Statistical characterization of efficiency, against a Stop\&Wait protocol}
  \item{Conclusion - Summary of the above descriptions, reflection on the learning objectives}
\end{enumerate}



\section{Architecture}

\section{Code Structure}

\section{Main use cases}

\section{The Data link Layer}

\section{The Application Layer}

\section{Validation}

\section{Efficiency}

\section{Conclusion}

\end{document}
